\chapter{PLANTEAMIENTO DEL PROBLEMA}
\label{chapter:problema}

\section{Descripción de la realidad problemática}
\label{section:description}
El café es una de las bebidas más consumida por las personas alrededor del mundo. Su sabor, acidez, aroma, cuerpo y consistencia son una mezcla que gusta a muchísimos. En Perú es uno de los principales productos agrícolas de exportación; sin embargo, en comparación con otros países productores, el rendimiento de producción es relativamente bajo~\citep{Diaz2017}.

Para llegar a obtener granos de calidad se debe tener en cuenta diferentes factores como: temperatura, altura, vientos, lluvias, humedad~\citep{Figueroa2015}, etc, algunos no serán considerados en este trabajo, aquí nos enfocamos en la calidad post-cosecha, donde la calidad de los granos de café está determinada por sus características físicas y organolépticas. Para preservar la calidad se debe llevar un control antes de la cosecha, durante la cosecha y después de la cosecha de los granos. En la fase de post-cosecha se da el beneficio húmedo, este proceso comprende varios pasos para conservar la calidad del producto como: recepción y pesado del cerezo, clasificación del cerezo, despulpado, fermentación, lavado, secado, almacenamiento del café y manejo de residuos del beneficio del café. Posterior al beneficio húmedo se da el beneficio seco, el cual ocurre antes de la exportación, está conformado por los pasos de: trillado, pulido y selección final, en esta selección final se hará la clasificación por tamaño, forma, densidad y color de los granos de café verde~\citep{Marin2013}.

En el proceso de selección de granos de café son considerados diferentes características o cualidades, entre ellos tenemos: el tamaño, aspecto del café, color, abertura de la hendidura, aroma, cuerpo, acidez, amargo y sabor~\citep{Lara2005}, para el proceso de control de calidad físico se utiliza una muestra de 350 gramos por lote. En este trabajo nos enfocaremos en la clasificación por tamaño de granos de café, para lo cual se usa actualmente tamices de entre 13 y 20 milímetros~\citep{Marin2013}. 

La clasificación por tamaños de los granos de café verde, es un parámetro de calidad, entonces aparece el término granulometría, el cual tiene como objetivo establecer una metodología de rutina para realizar el análisis del tamaño de grano entero del café verde mediante el tamizado mecánico, utilizando zarandas~\citep{Funez2010}. Para realizar esta tarea, se debe seleccionar tamices de acuerdo al tipo de clasificación que se desea obtener, los tamices deben estar ordenados de forma ascendente de acuerdo al número que poseen y ser colocados en un agitador, este zarandeo se debe hacer por lo menos dos veces sobre una muestra de 100 gramos, pero antes se debe separar materia extraña y granos partidos. Por cada zarandeada se debe obtener los resultados en términos de porcentaje de masa, en este sentido, la suma de cada uno de los tamices, más los objetos extraños y los granos en el fondo, debe ser igual a los 100 gramos iniciales, de lo contrario, se debe hacer nuevamente todo el proceso. Como resultado final se promedia las masas obtenidas por zaranda realizada para obtener los porcentajes por tamaños de los granos.

La homogeneidad que puede existir en el tamaño de los granos de café, es también un índice de calidad en los granos, esto nos lleva a enfocarnos en las características físicas de los granos de café verde, más precisamente en el tamaño de estos, para el proceso de selección de granos de calidad. Actualmente se utilizan tamices con diferentes diámetros y se hace una clasificación por porcentaje de granos de café, según los tamices.

El proceso manual realizado para este proceso es tedioso y trabajoso, dado que la persona debe estar moviendo los tamices, contabilizando y anotando los resultados, esto puede hacer que la persona a cargo de este proceso se canse rápidamente y en consecuencia realizar este paso de forma errónea, lo que podría provocar una deficiencia en todo el proceso de control de calidad de los granos de café verde.

%% ----------------------------------------------------------------------------------------------- %%

\section{Delimitaciones y definición del problema}

\subsection{Delimitaciones}
Se usará técnicas de procesamiento de imágenes digitales para realizar una clasificación de una muestra de granos de café verde en un ambiente controlado. Esta clasificación se hará en base a las medidas entre 13 y 20 milímetros, según el manual técnico de control de calidad del café. Se usará una cámara digital como dispositivo de adquisición. La imagen contará con una referencia en la escena la cual tendrá una medida establecida, esta se utilizará para obtener una relación cuantificable entre milímetros y píxeles. En este trabajo no consideramos escenas en movimiento ni factores externos al proceso de control de calidad de granos de café.

\subsection{Definición del problema}
El proceso manual para la clasificación por tamaño de granos de café verde es largo y tedioso para las personas a cargo de este proceso, el tamizado se debe hacer varias veces para obtener datos más precisos mediante el ponderado de los datos y si se hace mal, se debe volver hacer nuevamente todo el proceso, esto puede generar una deficiencia en el control de calidad. Es necesario automatizar este proceso de clasificación por tamaño de granos de café verde en laboratorios de control de calidad.

%% ----------------------------------------------------------------------------------------------- %%

\section{Problema principal}
La clasificación por tamaños de granos de café, se hace de forma manual en la actualidad, pero por ser un proceso largo y que necesita una buena precisión, se hará uso de técnicas de procesamiento de imágenes digitales para automatizar este proceso, de esta forma tener resultados más precisos de la clasificación por tamaños. Se considera la utilización de imágenes digitales de granos de café obtenida bajo condiciones de iluminación controlada, también se utilizará un objeto de referencia con medidas establecidas, el cual servirá para el cálculo de la relación pixel-milímetros, a las escenas digitales se les aplicará operadores morfológicos y de segmentación para hallar medidas estadísticas relevantes al proceso de tamizado. Finalmente, los resultados del diámetro promedio por grano serán evaluados y reportados.

%% ----------------------------------------------------------------------------------------------- %%

\section{Objetivos}
\label{section:objectives}

\subsection{Objetivo general}
Desarrollar un sistema computacional para la medición absoluta de diámetros de granos de café verde.

\subsection{Objetivos específicos}
\begin{itemize}
	\item Generar un conjunto de imágenes de granos de café verde clasificado por medidas.
	\item Elaborar un modelo de segmentación de granos de café en una imagen digital.
	\item Elaborar un modelo de análisis morfológico para la estimación de medidas.
	\item Analizar y validar el modelo computacional propuesto.
\end{itemize}

%% ----------------------------------------------------------------------------------------------- %%

\section{Hipótesis de la investigación}
Es probable desarrollar un sistema que pueda clasificar roubstamente una muestra de granos de café verde por diámetros mediante el uso de técnicas de procesamiento de imágenes digitales.

%% ----------------------------------------------------------------------------------------------- %%

\section{Variables e indicadores}

\subsection{Variable independiente}
En este trabajo se considera a la medida real de cada muestra como una variable independiente, luego se considera una variable que realiza la conversión de píxeles a milímetros, esta medida origina cambios en la medida obtenida y es controlada mediante un proceso de calibración manual, entonces es considera una variable independiente.

\subsubsection{Indicadores}
La medida real de cada muestra y nuestra variable de conversión se indican en milímetros.

\subsubsection{Índices}
La medida de cada muestra (MR) s	e indica en un rango de $[0,20]$ y $\text{MR} \in Z$, nuestra variable de conversión (C) se indica en un rango de $[0, \infty \rangle$ donde $C \in R$

\subsection{Variable dependiente}
Para este trabajo la medida obtenida de diámetro para cada muestra de café será nuestra variable dependiente.

\subsubsection{Indicadores}
Se considera un buen resultado cuando nuestra medida obtenida de diámetro es igual que la medida real, entonces un indicador es el error que existe entre ambas medidas, entre los posibles indicadores proponemos el uso del Error Mínimo Cuadrático o MSE.

\subsubsection{Índices}
El error (E) varía en un rango de $[0,+\infty\rangle$ considerando valores próximos a cero como resultados deseados y $E \in R$.

%% ----------------------------------------------------------------------------------------------- %%

\section{Viabilidad de la investigación.}

\subsection{Viabilidad técnica}
El presente proyecto es viable técnicamente dado que existen técnicas de procesamiento de imágenes digitales para el análisis de los granos de café, se han investigado diversas técnicas para que se pueda identificar a cada uno de los granos y clasificarlos de acuerdo a las medidas de los diámetros que posee cada uno.

También se cuenta con una cámara para la adquisición de imágenes y un conjunto de muestras evaluadas de granos de café de diferentes tamaños para poder realizar pruebas y evaluaciones del modelo a desarrollarse.

\subsection{Viabilidad operativa}
Para el desarrollo de este proyecto, el autor tiene los conocimientos básicos de las diferentes técnicas que se utilizarán durante todo el desarrollo del proyecto y cuenta con la guía de un experimentado asesor especializado en el análisis de imágenes por visión computacional, quién a su vez ha hecho trabajos relacionados a la calidad de los granos de café.

\subsection{Viabilidad económica}
Este proyecto es viable económicamente porque se tiene los recursos necesarios como una cámara fotográfica, una computadora para procesamiento y muestras de granos de café.  Se dispone de las licencias de software necesarias para la integración del modelo computacional y la cámara de adquisición. 

%% ----------------------------------------------------------------------------------------------- %%

\section{Justificación e importancia de la investigación.}

\subsection{Justificación}
En el proceso de control de calidad de granos de café verde, está incluido la clasificación por tamaños en una muestra, este proceso puede llegar a ser muy trabajoso y pesado para las personas dado que deben: primero pesar 100 gramos con una aproximación de 0.1 gramos de error de la muestra, registrar la presencia de materia extraña y granos partidos, seleccionar lo tamices a utilizar de acuerdo al tipo de clasificación que se desea, colocando las mallas de menos diámetro en la parte inferior y luego colocar los 100 gramos de la prueba en la parte superior, agitar o programar a la máquina para que agite por el lapso de dos minutos, pesar los granos retenidos en cada malla y repetir todo el proceso por lo menos una vez más. Luego se debe contabilizar y registrar los resultados, la sumatoria de los granos en cada malla más los defectuosos debe ser igual a los 100 gramos iniciales con aproximación a 0.5 gramos, caso contrario, la prueba no será válida y se deberá repetir el proceso con una nueva muestra~\citep{Funez2010}. Esta etapa del proceso es muy importante, porque una gran cantidad de granos pequeños perjudicarían posteriormente el tueste de los granos grandes y viceversa, provocando que los granos pequeños se tuesten demasiado en el tiempo que requieren tostarse los granos grandes o que los granos grandes no alcanzarán el tueste necesario en el tiempo que requieren los granos pequeños~\citep{Hilten2011}. Es necesario automatizar este proceso buscando disminuir el tiempo requerido para esta tarea, también disminuir la carga de trabajo realizada por el personal permitiéndole enfocarse en tareas de control de calidad más importantes como la catación de café.

\subsection{Importancia}
Como se ha expuesto, todo el proceso de selección de granos de café de calidad está conformado por distintos pasos, de los cuales, el presente trabajo se centra en la clasificación de una muestra de granos de café verde, de acuerdo al tamaño de su diámetro. Para realizar dicha tarea se propone un modelo computacional que se encargará de realizar este paso de forma automática mediante la adquisición y procesado de imágenes fotográficas digitales de la muestra de granos de café.

De esta forma se busca tener una mejor precisión en la clasificación de los granos de café en una muestra, por lo que el catador se centrará en el análisis de los datos obtenidos y no en realizar todo el proceso adecuadamente, como se expuso anteriormente, elevando así la precisión de medición de la muestra, por lo que la probabilidad de un tostado uniforme de todos los granos, en los siguientes pasos de todo el proceso, sea más elevada.

%% ----------------------------------------------------------------------------------------------- %%

\section{Alcance}
Se pretende automatizar la clasificación por diámetro de granos de café verde en una muestra, por medio del desarrollo de un modelo computacional que sea capaz de analizar imágenes digitales  de granos de café para según los tamaños clasificarlos.

%% ----------------------------------------------------------------------------------------------- %%

\section{Tipo y nivel de la investigación}

\subsection{Tipo de la investigación}
Este proyecto pertenece a una investigación de tipo aplicada, ya que se quiere resolver un problema de la vida real mediante el procesamiento de imágenes digitales ya desarrolladas para la resolución de nuestro problema.

\subsection{Nivel de la investigación}
El nivel de la investigación de este proyecto es aplicativo, porque se usarán métodos y técnicas de procesamiento de imágenes para la extracción de medidas de los granos de café verde.

%% ----------------------------------------------------------------------------------------------- %%

\section{Método y diseño de la investigación}

\subsection{Método de la investigación}
El método de investigación de este proyecto de inductivo, debido a que se hará un análisis de los resultados obtenidos a partir de los experimentos de medición que se harán con el sistema propuesto.

\subsection{Diseño de la investigación}
El diseño de la presente investigación es de nivel exploratoria, ya que a partir del problema planteado, y siguiendo un conjunto de etapas, se usarán algunas técnicas de procesamiento de imágenes digitales para poder alcanzar el objetivo planteado disminuyendo así la carga de trabajo de los catadores.

%% ----------------------------------------------------------------------------------------------- %%

\section{Técnicas e instrumentos de recolección de información}

\subsection{Técnicas}
................

\subsection{Instrumentos}
................

%% ----------------------------------------------------------------------------------------------- %%

\section{Cobertura de estudio}

\subsection{Universo}
En el proceso de control de calidad el proceso de tamizado se aplica a todos los tipos de granos de café reconocidos por el estándar mundial de control de calidad de granos de café,  existen dos especies: arábiga y robusta, se plantea trabajar un modelo de clasificación de granos de café por diámetro aplicable a todas las variedades de granos.

\subsection{Muestra}
Para este trabajo se considera el uso de muestras físicamente evaluadas por catadores especializados equivalente a 1 kg, estas muestras fueron clasificadas utilizando zarandas de 13 a 20mm.

%% ----------------------------------------------------------------------------------------------- %%

\section{Cronograma y presupuesto}

\subsection{Universo}
................

\subsection{Muestra}
................

%% ----------------------------------------------------------------------------------------------- %%

\section{Estructura del documento}
...................